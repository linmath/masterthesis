\chapter{Introduction}

The classical Riemann-Hilbert correspondence asserts the existence of a functor mapping flat connections on algebraic vector bundles over an algebraic variety $X$ with regular singularities to the category of local systems of finite-dimensional complex vector spaces on $X$. 
Generalizing this to connections with irregular singularities raises the question of how to encode the ``extra data'' to the local systems. This can be done by using Stokes filtrations (shown in \cite{Sabbah_StokesStructures}) or by Stokes matrices. The notion of Stokes shells, invented by T. Mochizuki in \cite{mochistokes}, can also be used to describe the data of the irregular singularity.
\newline

In this work we will study differential systems of pure Gaussian type, which are modules with connections on the projective line that have only one singularity: an irregular one located at infinity. Using the Riemann-Hilbert correspondence for differential systems that are of pure Gaussian type, we can study these systems from an algebraic-topological point of view. Specifically, we shift our focus towards investigating Stokes filtered local systems and - the more classical approach - Stokes data of Gaussian type. 

Our primary objective is to introduce and define the concept of Stokes shells within the context of Gaussian type differential systems. Furthermore, we aim to establish an equivalence between the category of Stokes shells and the category of Stokes data. 

To provide some context, we will begin with a brief review of local systems. Then, in Chapter 3, we will introduce a formal definition of differential systems of pure Gaussian type. Furthermore, after studying Stokes filtrations and Stokes data within our framework, we will formulate the concept of Stokes shells within the context of Gaussian type systems in Chapter 4. Ultimately, in Chapter 5, we establish an equivalence of categories, bridging the world of Stokes data with that of Stokes shells, particularly focusing on a specific class of differential systems of Gaussian type.
\newline

The reader is supposed to be familiar with basic sheaf and category theory, as well as commutative algebra.
